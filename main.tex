%% abtex2-modelo-trabalho-academico.tex, v-1.9.2 laurocesar
%% Copyright 2012-2014 by abnTeX2 group at http://abntex2.googlecode.com/ 
%%
%% This work may be distributed and/or modified under the
%% conditions of the LaTeX Project Public License, either version 1.3
%% of this license or (at your option) any later version.
%% The latest version of this license is in
%%   http://www.latex-project.org/lppl.txt
%% and version 1.3 or later is part of all distributions of LaTeX
%% version 2005/12/01 or later.
%%
%% This work has the LPPL maintenance status `maintained'.
%% 
%% The Current Maintainer of this work is the abnTeX2 team, led
%% by Lauro César Araujo. Further information are available on 
%% http://abntex2.googlecode.com/
%%
%% This work consists of the files abntex2-modelo-trabalho-academico.tex,
%% abntex2-modelo-include-comandos and abntex2-modelo-references.bib
%%

% ------------------------------------------------------------------------
% ------------------------------------------------------------------------
% abnTeX2: Modelo de Trabalho Academico (tese de doutorado, dissertacao de
% mestrado e trabalhos monograficos em geral) em conformidade com 
% ABNT NBR 14724:2011: Informacao e documentacao - Trabalhos academicos -
% Apresentacao
% ------------------------------------------------------------------------
% ------------------------------------------------------------------------

% ------------------------------------------------------------------------
% ------------------------------------------------------------------------
% Modelo modificado em 08/08/2022 por Gabriel Pietroluongo a fim de se
% adequar ao formato de relatório de estágio obrigatório da UFES 
% (Universidade Federal do Espírito Santo). 
% Você pode me encontrar como @pietroluongo no GitHub ;)
% ------------------------------------------------------------------------
% ------------------------------------------------------------------------

\documentclass[
	% -- opções da classe memoir --
	12pt,				% tamanho da fonte
	openright,			% capítulos começam em pág ímpar (insere página vazia caso preciso)
	oneside,			% para impressão em verso e anverso. Oposto a oneside
	a4paper,			% tamanho do papel. 
	% -- opções da classe abntex2 --
	%chapter=TITLE,		% títulos de capítulos convertidos em letras maiúsculas
	%section=TITLE,		% títulos de seções convertidos em letras maiúsculas
	%subsection=TITLE,	% títulos de subseções convertidos em letras maiúsculas
	%subsubsection=TITLE,% títulos de subsubseções convertidos em letras maiúsculas
	% -- opções do pacote babel --
	english,			% idioma adicional para hifenização
	french,				% idioma adicional para hifenização
	spanish,			% idioma adicional para hifenização
	brazil				% o último idioma é o principal do documento
	]{abntex2}

% ---
% Pacotes básicos 
% ---
\usepackage{lmodern}			% Usa a fonte Latin Modern			
\usepackage[T1]{fontenc}		% Selecao de codigos de fonte.
\usepackage[utf8]{inputenc}		% Codificacao do documento (conversão automática dos acentos)
\usepackage{lastpage}			% Usado pela Ficha catalográfica
\usepackage{indentfirst}		% Indenta o primeiro parágrafo de cada seção.
\usepackage{color}				% Controle das cores
\usepackage{graphicx}			% Inclusão de gráficos
\usepackage{microtype} 			% para melhorias de justificação
\usepackage[clean]{svg}
% ---
		
% ---
% Pacotes adicionais, usados apenas no âmbito do Modelo Canônico do abnteX2
% ---
\usepackage{lipsum}				% para geração de dummy text
% ---

% ---
% Pacotes de citações
% ---
\usepackage[brazilian,hyperpageref]{backref}	 % Paginas com as citações na bibl
\usepackage[alf]{abntex2cite}	% Citações padrão ABNT

% --- 
% CONFIGURAÇÕES DE PACOTES
% --- 

% ---
% Configurações do pacote backref
% Usado sem a opção hyperpageref de backref
\renewcommand{\backrefpagesname}{Citado na(s) página(s):~}
% Texto padrão antes do número das páginas
\renewcommand{\backref}{}
% Define os textos da citação
\renewcommand*{\backrefalt}[4]{
	\ifcase #1 %
		Nenhuma citação no texto.%
	\or
		Citado na página #2.%
	\else
		Citado #1 vezes nas páginas #2.%
	\fi}%
% ---

% ---
% Informações de dados para CAPA e FOLHA DE ROSTO
% ---
\titulo{Relatório Final de Estágio Supervisionado}
\autor{Nome do aluno}
\orientador{Nome do Supervisor}
\coorientador{Nome do coordenador}
\data{Agosto de 2022}
\instituicao{UNIVERSIDADE FEDERAL DO ESPÍRITO SANTO\\CENTRO TECNOLÓGICO\\CURSO DE ENGENHARIA DE COMPUTAÇÃO}
\tipotrabalho{Tese (Doutorado)}
\local{Vitória}
% O preambulo deve conter o tipo do trabalho, o objetivo, 
% o nome da instituição e a área de concentração 
\preambulo{Relatório Final de Estágio apresentado à UFES como requisito parcial da disciplina de Estágio Supervisionado do Curso de Engenharia de Computação.}
% ---


% ---
% Configurações de aparência do PDF final

% alterando o aspecto da cor azul
\definecolor{blue}{RGB}{41,5,195}

% informações do PDF
\makeatletter
\hypersetup{
     	%pagebackref=true,
		pdftitle={\@title}, 
		pdfauthor={\@author},
    	pdfsubject={\imprimirpreambulo},
	    pdfcreator={LaTeX with abnTeX2},
		pdfkeywords={abnt}{latex}{abntex}{abntex2}{trabalho acadêmico}, 
		colorlinks=true,       		% false: boxed links; true: colored links
    	linkcolor=blue,          	% color of internal links
    	citecolor=blue,        		% color of links to bibliography
    	filecolor=magenta,      		% color of file links
		urlcolor=blue,
		bookmarksdepth=4
}
\makeatother
% --- 

% --- 
% Espaçamentos entre linhas e parágrafos 
% --- 

% O tamanho do parágrafo é dado por:
\setlength{\parindent}{1.3cm}

% Controle do espaçamento entre um parágrafo e outro:
\setlength{\parskip}{0.2cm}  % tente também \onelineskip

% ---
% compila o indice
% ---
\makeindex
% ---

% ----
% Início do documento
% ----
\begin{document}

% Retira espaço extra obsoleto entre as frases.
\frenchspacing 

% ----------------------------------------------------------
% ELEMENTOS PRÉ-TEXTUAIS
% ----------------------------------------------------------
% \pretextual

\makeatletter
\renewcommand{\folhaderostocontent}{
\begin{center}
\begin{figure}
    \centering
    \includegraphics[width=0.4\columnwidth]{brasaooficialcolorido.png}
\end{figure}
\vspace*{}
{\ABNTEXchapterfont\large\imprimirinstituicao}
\vspace*{\fill}\vspace*{\fill}


{\ABNTEXchapterfont\bfseries\Large\imprimirtitulo}


\vspace*{\fill}
\vspace*{\fill}
\begin{flushright}
\abntex@ifnotempty{\imprimirpreambulo}{%
\hspace{.45\textwidth}
\begin{minipage}{.5\textwidth}
\SingleSpacing
\imprimirpreambulo
\end{minipage}%
\end{flushright}
\vspace*{\fill}
}%
\vspace*{\fill}
{\imprimirlocal, \large\imprimirdata}
\vspace*{1cm}
\end{center}
}
\makeatother


% ---
% Folha de rosto
% (o * indica que haverá a ficha bibliográfica)
% ---
\folhaderostocontent
% ---
% ---
% ---

% ---
% Inserir folha de aprovação
% ---

% Isto é um exemplo de Folha de aprovação, elemento obrigatório da NBR
% 14724/2011 (seção 4.2.1.3). Você pode utilizar este modelo até a aprovação
% do trabalho. Após isso, substitua todo o conteúdo deste arquivo por uma
% imagem da página assinada pela banca com o comando abaixo:
%
% \includepdf{folhadeaprovacao_final.pdf}
%
\begin{folhadeaprovacao}

  \begin{center}
    {\ABNTEXchapterfont\large\imprimirautor}

    \vspace*{\fill}\vspace*{\fill}
    \begin{center}
      \ABNTEXchapterfont\bfseries\Large\imprimirtitulo
    \end{center}
    \vspace*{\fill}
    
    \hspace{.45\textwidth}
    \begin{minipage}{.5\textwidth}
        \imprimirpreambulo
    \end{minipage}%
    \vspace*{\fill}
   \end{center}
   
   \begin{flushright}
       

   \assinatura*{\textbf{\imprimirorientador} \\ Supervisor na Empresa Nome da Empresa} 
   \assinatura*{\textbf{\imprimircoorientador} \\ Coordenador de Estágio na UFES}
   \assinatura*{\textbf{\imprimirautor} \\ Estagiário}
   
   \end{flushright}
      
   \begin{center}
    \vspace*{0.5cm}
    Vitória,
    {\large\imprimirdata}
    \vspace*{1cm}
  \end{center}
  
\end{folhadeaprovacao}


% ---
% inserir o sumario
% ---
\pdfbookmark[0]{\contentsname}{toc}
\tableofcontents*
\cleardoublepage
% ---

% ---

% ---
% inserir lista de ilustrações
% ---
\pdfbookmark[0]{Lista de Figuras}{lof}
\listoffigures*
Elemento opcional, elaborado de acordo com a ordem apresentada no texto, com
cada item designado por seu nome específico, acompanhado do respectivo número da
página. Recomenda-se a elaboração de lista própria para cada tipo de ilustração (desenhos,
fluxogramas, fotografias, gráficos, mapas, organogramas, plantas, quadros, retratos e
outros).
\cleardoublepage
% ---

% ---
% inserir lista de tabelas
% ---
\pdfbookmark[0]{\listtablename}{lot}
\listoftables*
Elemento opcional, elaborado de acordo com a ordem apresentada no texto, com
cada item designado por seu nome específico, acompanhado do respectivo número da
página.
\cleardoublepage
% ---

% ---
% inserir lista de abreviaturas e siglas
% ---
\begin{siglas}
  \item[ABNT] Associação Brasileira de Normas Técnicas
  \item[abnTeX] ABsurdas Normas para TeX
    \item Elemento opcional, constituída de uma relação alfabética das abreviaturas e siglas utilizadas no texto, seguido das palavras ou expressões correspondentes grafadas por extenso. Quando necessário, recomenda-se a elaboração de lista própria para cada tipo.
\end{siglas}
% ---

% ---
% inserir lista de símbolos
% ---
\begin{simbolos}
  \item[$ \Gamma $] Letra grega Gama
  \item[$ \Lambda $] Lambda
  \item[$ \zeta $] Letra grega minúscula zeta
  \item[$ \in $] Pertence
  \item Elemento opcional elaborado de acordo com a ordem apresentada no texto, seguido
do significado correspondente.
\end{simbolos}
% ---


% ----------------------------------------------------------
% ELEMENTOS TEXTUAIS
% ----------------------------------------------------------
\textual

% ----------------------------------------------------------
% Introdução (exemplo de capítulo sem numeração, mas presente no Sumário)
% ----------------------------------------------------------
\chapter[Plano de estágio]{Plano de estágio}
% ----------------------------------------------------------
\section{Identificação do Aluno}

\parindent=0pt

Nome:\imprimirautor

Código do Aluno na UFES:

E-mail:

\section{Empresa}
Nome:

Razão Social:

CGC:

Área de atuação:

Endereço:

Bairro:

CEP:

Cidade:

Estado:

Nome do responsável pelos estágios na empresa (Recursos Humanos, Recrutamento e Seleção etc.):

Telefone da área responsável pelos estágios:

\section{Estágio}
Área de atuação:

Setor:

Descrição geral:

Organograma: (indique, através de representação gráfica, a sua posição como estagiário dentro do organograma da empresa).

Data de início do estágio:

Data prevista para o fim do estágio:

Período do dia em que estagia:

Carga horária semanal:
\section{Supervisor de Estágio na empresa}
Nome:

Formação acadêmica na graduação:

Cargo:

Departamento ou setor que trabalha:

Responsabilidades do departamento ou setor que trabalha:

Telefone:

Fax:

E-mail:

\section{Atividades Programadas para o Estagiário}
(Descreva o programa de trabalho do seu estágio, incluindo atividades e os resultados esperados.)

\chapter[Organograma da Empresa]{Organograma da Empresa}

\section{A empresa}
(Faça um histórico sobre a empresa, indicando o ano de fundação e o número de
funcionário).

\section{Objeto de Produção da Empresa e Missão}
(Apresente o que a empresa produz e a missão da mesma).

\section{Organograma geral}
(Indique, através de representação gráfica, o organograma da empresa).

\section{Organograma Específico do Setor do Estágio}
(Indique, através de representação gráfica, a sua posição como estagiário dentro do
organograma da empresa).

\section{Atribuições do Setor onde foi Desenvolvido o Estágio}
(Indique o que é feito no setor onde fez o estágio).

\section{Atribuições do Setor onde foi Desenvolvido o Estágio}
(Indique o que é feito no setor onde fez o estágio).

\section{Processo de Seleção para o Estágio}
(Indique como foi o processo de seleção para o estágio).

\chapter{Recursos disponíveis para a Realização do Estágio}

\section{Ambiente de Trabalho}
(Descreva o ambiente de trabalho da empresa).

\section{Maquinas, Equipamentos e Softwares Utilizados}
(Descreva tudo o que foi utilizado durante o seu trabalho na empresa).

\section{Oficinas e Laboratórios}
(Indique que oficinas e laboratórios utilizou na empresa).

\section{Equipe de Trabalho}
(Indique quantos foram e quais tarefas tiveram os integrantes da sua equipe de
trabalho).

\section{Inter-relação com Outras Áreas da Empresa}
(Fale do seu relacionamento com outras áreas da empresa, tanto as técnicas como
as administrativas).

\chapter{Atribuições do estagiário}
\section{Grau de Autonomia e Responsabilidade}
(Fale sobre o grau de autonomia e a responsabilidade que a empresa lhe atribuiu).

\section{Forma de Planejamento e Organização das Tarefas}
(Indique como o seu supervisor planejou e organizou as tarefas para o seu estágio).

\section{Forma de Executar e Implementar as Tarefas}
(Indique os métodos utilizados para executar e implementar as tarefas do seu
estágio).
\section{Procedimento de Controle, Itens de Verificação e Indicadores de
Desempenho}
(Fale que instrumentos a empresa utilizou para verificar o seu desempenho).

\section{Forma de Acompanhamento e Avaliação dos Resultados pelo
Supervisor}
(Fale como a empresa acompanhou a execução das suas tarefas).


\chapter{Atividades desenvolvidas}
\section{Atividades Desenvolvidas e Resultados Obtidos}

(Como, muitas vezes, as atividades planejadas nem sempre são as desenvolvidas,
descreva as atividades realmente desenvolvidas e cronograma real para sua execução.
Indique também os resultados obtidos e o número de horas para cada tarefa. Acrescente
mais linhas se precisar).

\chapter{Comentários do Estagiário}
\section{Adaptação ao Método de Trabalho da Empresa}
(Indique como foi a sua adaptação ao método de trabalho da empresa).

\section{Oportunidades de Desenvolvimento Profissional}
(Fale sobre as oportunidades de desenvolvimento pessoal que a empresa lhe ofereceu e pode lhe oferecer no futuro).

\section{Dificuldades Encontradas}
(Identifique e apresenta as dificuldades encontradas durante o estágio).

\section{Sugestões de Melhoria em Procedimento de Trabalho Observados}
(Em função das dificuldades encontradas durante o estágio, faça algumas sugestões que poderiam melhorar o procedimento de trabalho adotado pela empresa no seu caso).

\section*{Conclusões}
\addcontentsline{toc}{chapter}{Conclusões}
(Identifique e apresente os prós e contras do seu o estágio e, a partir disso, indique a sua impressão positiva ou negativa sobre o mesmo).

% ----------------------------------------------------------
% PARTE
% ----------------------------------------------------------
\chapter{Referências Bibliográficas}
% ----------------------------------------------------------


% ----------------------------------------------------------
% Finaliza a parte no bookmark do PDF
% para que se inicie o bookmark na raiz
% e adiciona espaço de parte no Sumário
% ----------------------------------------------------------
%\phantompart


% ----------------------------------------------------------
% ELEMENTOS PÓS-TEXTUAIS
% ----------------------------------------------------------
\postextual
% ----------------------------------------------------------

% ----------------------------------------------------------
% Referências bibliográficas
% ----------------------------------------------------------
\bibliography{abntex2-modelo-references}

% ----------------------------------------------------------
% Glossário
% ----------------------------------------------------------
%
% Consulte o manual da classe abntex2 para orientações sobre o glossário.
%
%\glossary

% ----------------------------------------------------------
% Apêndices
% ----------------------------------------------------------

% ---
% Inicia os apêndices
% ---
\begin{apendicesenv}

% Imprime uma página indicando o início dos apêndices
% \partapendices

% ----------------------------------------------------------
\chapter{Título do Apêndice}
% ----------------------------------------------------------

Consiste em texto ou documento elaborado pelo autor, a fim de complementar sua argumentação, sem prejuízo da unidade nuclear do trabalho. Os apêndices devem ser identificados por letras maiúsculas consecutivas, seguidas de travessão e respectivo título.

Inclua aqui os Relatórios Parciais, se for o caso.

\end{apendicesenv}
% ---


% ----------------------------------------------------------
% Anexos
% ----------------------------------------------------------

% ---
% Inicia os anexos
% ---
\begin{anexosenv}

% Imprime uma página indicando o início dos anexos
% \partanexos

% ---
\chapter{Título do anexo}
% ---
Consiste em texto ou documento não elaborado pelo autor, que serve de fundamentação, comprovação e ilustração. Os anexos devem ser identificados por letras maiúsculas consecutivas, seguidas de travessão e respectivo título. Inclua aqui as fichas de avaliação do Supervisor de Estágio e do Professor Orientador de Estágio.

\end{anexosenv}

%---------------------------------------------------------------------
% INDICE REMISSIVO
%---------------------------------------------------------------------
%\phantompart
\printindex
%---------------------------------------------------------------------

\end{document}
